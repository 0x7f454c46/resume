% Resume of Cies Breijs
% =====================

% On to generate a pdf from this file in Ubuntu:
%   sudo aptitude install texlive-xetex tex-gyre
%   xelatex cies-breijs-resume.tex
%
% I don't use pdflatex because XeTeX is the future...

% vim:set ft=tex spell:

\documentclass[10pt,a4paper]{article}
\usepackage[a4paper,margin=0.75in]{geometry}
\usepackage[utf8]{inputenc}
\usepackage{mdwlist}
\usepackage{multicol}
\usepackage{epsfig}
\usepackage[T1]{fontenc}
\usepackage{textcomp}
\usepackage{wrapfig}
\usepackage{relsize}  % for \textscale, which I prefer over \sc (small caps), see my \acr command
\usepackage[pdftex]{hyperref}  % yups, URLs everwhere...
\usepackage{xcolor}  % color them links
\definecolor{dark-blue}{rgb}{0.15,0.15,0.4}
\hypersetup{colorlinks,linkcolor={dark-blue},citecolor={dark-blue},urlcolor={dark-blue}}
\usepackage{lettrine}
\usepackage{tgpagella}  % the pretty fonts
\usepackage{fontspec}
\addfontfeature{Style=Historic}
\setmainfont
  [ ExternalLocation ,
    Mapping = tex-text ,
    Numbers = OldStyle ,
    Ligatures= {Common,Historical,Contextual,Rare} ,
    BoldFont = texgyrepagella-bold.otf ,
    ItalicFont = texgyrepagella-italic.otf ,
    BoldItalicFont = texgyrepagella-bolditalic.otf ]
  {texgyrepagella-regular.otf}

% use this for the libertine font (has more lignatures), install 'ttf-linux-libertine' on ubuntu
% \setmainfont
%   [ ExternalLocation = /usr/share/fonts/truetype/linux-libertine/ ,
%     Mapping = tex-text ,
%     Numbers = OldStyle ,
%     Ligatures= {Common,Historical,Contextual,Rare} ,
%     BoldFont = LinLibertine_Bd.ttf ,
%     ItalicFont = LinLibertine_It.ttf ,
%     BoldItalicFont = LinLibertine_BI.ttf ]
%   {LinLibertine_Re.ttf}

% document wide styling
\pagestyle{empty}
\setlength{\tabcolsep}{0em}
\xspaceskip7pt  % some more spacing between sentences (use "i.e.\ " or "with SQL\@. " in case of errors)


% capnums command, for new/lining/capital numbers
\newcommand*\capnums[1]{{\fontencoding{T1}\fontfamily{pplx}\selectfont #1}}

% acr command, to quickly mark acronyms for special formatting
\newcommand*\acr[1]{\textscale{.85}{#1}}

% indentsection style, used for sections that aren't already in lists
% that need indentation to the level of all text in the document
\newenvironment{indentsection}[1]%
{\begin{list}{}%
  {\setlength{\leftmargin}{#1}}%
  \item[]%
}
{\end{list}}

% opposite of above; bump a section back toward the left margin
\newenvironment{unindentsection}[1]%
{\begin{list}{}%
  {\setlength{\leftmargin}{-0.5#1}}%
  \item[]%
}
{\end{list}}

% headerrow command, used for a new employer
\newcommand{\headerrow}[2]
{\begin{tabular*}{\linewidth}{l@{\extracolsep{\fill}}r}
  \textscale{1.08}{\textbf{#1}} &
  {#2} \\
\end{tabular*}}

% subheaderrow command, used for a new position
\newcommand{\subheaderrow}[2]
{\begin{tabular*}{\linewidth}{l@{\extracolsep{\fill}}r}
  \emph{#1} &
  \emph{#2} \\
\end{tabular*}}

% CPP command, make "C++" look pretty when used in text by touching up the plus signs
\newcommand{\CPP}{C\nolinebreak[4]\hspace{-.04em}\raisebox{.20ex}{\footnotesize\bf ++}}

% KTurtle command, make the document a bit more readable
\newcommand{\KTurtle}{\acr{KT}urtle }

% apo command, for an apostrophe that looks good on old style nums
\newcommand{\apo}{\raisebox{-.18ex}{'}{\hspace{-.1em}}}



%% some stuff to the pictures to work...
% \usepackage[absolute,showboxes]{textpos}
% \usepackage[colorgrid,texcoord]{eso-pic}
%
%%adjust the TPHorizModule and TPHorizModule units to the displayed mm %grid
% \TPGrid{210}{297}
%
%%puts a graphic at the absolute position described by the grid
%%#1 x, #2 y, #3 width, #4 height, #5 graphic
% \newcommand\putpic[5]{%
        % \begin{textblock}{#3}(#1,#2)
  % \includegraphics[width=#3\TPHorizModule,
  % height=#4\TPVertModule]{#5}
     % \end{textblock}
% }



\begin{document}

% \putpic{162.5}{13.5}{23}{23}{cies-face}

\hspace{-\parindent}{\LARGE \textbf{Cies Breijs}}\ \ \ \emph{June 12, 1982}

\vspace{0.3em}
\hspace{-\parindent}\href{mailto:cies at kde.nl}{cies$\,\!$@$\,\!$kde.nl}\ \ \textbullet  % email address obfuscated
\ \ \textsmaller{+}31.646469087\ \ \textbullet
\ \ \textsmaller{+}31.10.4226281\ \ \textbullet
\ \ \href{http://www.linkedin.com/in/ciesbreijs}{www.linkedin.com/in/ciesbreijs}
\\
Akkerwindestraat 1\ \ \textbullet
\ \ 3051\thinspace {\sc la}\ \ \textbullet
\ \ Rotterdam\ \ \textbullet
\ \ The Netherlands
\vspace{0.9em}


\hrule \vspace{-0.4em} \subsection*{Summary}

\vspace{-1.1em}
\begin{multicols}{2}
\noindent \emph{Creative software engineer with roots in the open source (\acr{FLOSS}) movement, an entrepreneurial mindset and a passion for user/customer centered software development.}
\\
\lettrine[lines=2,loversize=0.1,findent=-5pt]{A}{t the age} of seven (1989) I wrote my first lines using a \acr{LOGO}-like language on an \acr{MSX} (pre-\acr{PC}). Two years later I attended a conference on an emerging new technology, the Internet, at the Erasmus University from which I would graduate 16 years later.

After being introduced to the open source movement in 1997, I taught myself a variety of skills with help from the open source community. Since 2002 I am a contributor to the \acr{KDE} project, as my pet project \KTurtle got admitted into their \emph{edu} module.

\KTurtle aims to make programming most easy and touchable. It is well suited for teaching (children) the basics of programming, math and geometry. Through this project I hope to give others the same opportunity I was given:\ to learn programming at early age.

From 2003 to 2007 I studied at the Erasmus University Rotterdam and graduated in \emph{Business and Computer Science} (one curriculum). In college I mainly focused on the economics of open source, rapid application development (\acr{RAD}) and the semantic web technology stack (\acr{RDF}/\acr{RDFS}, \acr{OWL} and \acr{SPARQL}).

After graduation I traveled Europe and Asia during a two year sabbatical, I worked on several occasions (see experience below) and initiated an open source project (Truetopia -- a web application that facilitates self-governing online communities).

I am currently evaluating the plethora of interesting employment and business opportunities in \acr{IT} while quietly working on a product to be released later this year.
\end{multicols}


\hrule \vspace{-0.4em} \subsection*{Experience}

\begin{itemize} \parskip=0.1em

  \item  % Intellecap
  \headerrow
    {\href{http://www.intellecap.com}{Intellecap}}
    {\sc Mumbai \& Hyderabad, India}
  \\
  \subheaderrow
    {\acr{IT} Consultant}
    {Nov \apo08 -- Feb \apo09}
  {Intellecap is a social-sector advisory firm serving corporates, non-profits, development agencies and governments working in developing markets. I assessed their software development team and methodologies, trained their developers and build several web applications. One of those apps is Mostfit, an open source \acr{MIS} for \href{http://en.wikipedia.org/wiki/Microcredit}{microcredit} lenders.
  \vspace{0.2em}}
  \\
  \subheaderrow
    {\acr{IT} \& Strategy Consultant}
    {Jan \apo10 -- present}
  Mostfit became a success and I was called in to solve several technical challenges, look at potential growth strategies, and assist in making a business case for this product in a company of its own.

  \item  % Zarafa
  \headerrow
    {\href{http://www.zarafa.com}{Zarafa}}
    {\sc Delft, The Netherlands}
  \\
  \subheaderrow
    {\acr{QA} \& Release Manager}
    {Dec \apo09 -- Jan \apo11}
  Zarafa might be the fastest growing open source product company in Europe. For an open source enthusiast like myself, working with Zarafa is a wonderful opportunity. Reporting directly to the \acr{CEO}, Brian Josef, and working closely with the \acr{CTO}, Steve Hardy, I drove the change to test automation and continuous integration. Furthermore I architected new documentation and translation systems. My Indian work experiences proved particularly valuable, as I was sent to analyze and streamline their outsourced operations in India.

  \item  % Dharma Publishing
  \headerrow
    {\href{http://www.dharmapublishing.com}{Dharma Publishing}}
    {\sc near San Francisco (\acr{CA}), \acr{USA}}
  \\
  \subheaderrow
    {\acr{IT} Consultant}
    {Nov \apo09 -- Dec \apo09}
  Dharma Publishing, the worlds largest Buddhist publisher, is a non-profit, all-volunteer organization that helps to preserve Tibetan Buddhism and culture. I build them a \href{http://www.dharmapublishing.com}{web store}, and moved the infrastructure of all their web media to a nearly zero maintenance setup that they can maintain themselves.

  \item  % KDE
  \headerrow
    {\href{http://www.kde.org}{KDE}}
    {\href{http://edu.kde.org/kturtle}{edu.kde.org/kturtle}}
  \\
  \subheaderrow
    {Software Engineer}
    {Dec \apo03 -- present}
  \KTurtle is an educational programming environment that simplifies learning the basics of programming. At early age I learned programming using \acr{MSX-LOGO}, by creating \KTurtle I want to ensure future generation's access to a similar piece of software. I was honored when \KTurtle got admitted to \acr{KDE} in 2003.

  \item  % Truetopia Project
  \headerrow
    {\href{http://truetopiaproject.org}{Truetopia Project}}
    {\href{http://truetopiaproject.org}{truetopiaproject.org}}
  \\
  \subheaderrow
    {Initiator}
    {Nov \apo07 -- Apr \apo10}
  The Truetopia Project is an open source web application (Rails) to facilitate self-governing communities. It provides a workflow for collaborative problem identification and solution design.

\pagebreak

  \item  % DPU
  \headerrow
    {\href{http://www.dpu.ac.th/dpuic}{Dhurakij Pundit University International College}}
    {\sc Bangkok, Thailand}
  \\
  \subheaderrow
    {Guest Lecturer}
    {Sep \apo09}
  Dr.\@ Pilun Piyasirivej and Mr.\@ Michel Bauwens invited me to give two guest lectures:\ one on the open source movement and one on the semantic web.

  \item  % Opendream
  \headerrow
    {\href{http://www.opendream.th}{Opendream}}
    {\sc Bangkok, Thailand}
  \\
  \subheaderrow
    {\acr{IT} Consultant}
    {Aug \apo09 -- Sep \apo09}
  Architected and largely implemented an open source media sharing web service (\acr{REST} api) that facilitates video uploads, transcoding and streaming. Coached their development team on system design, ruby development (using Merb/Rails) and testing strategies such as \acr{TDD}/\acr{BDD}.

  \item  % Commuun
  \headerrow
    {\href{http://www.commuun.nl}{Commuun}}
    {\sc Rotterdam, The Netherlands}
  \\
  \subheaderrow
    {Senior Visionary}
    {Jul \apo06 -- Sep \apo09}
  I helped Peter Duijnstee (the proprietor of Commuun) with setting up the technical infrastructure, defining the core competences and creating a brand for his internet company. We collaborated on several web applications (all Rails apps) within the context of his company.

  \item  % EUR
  \headerrow
    {\href{http://www.eur.nl}{Erasmus University Rotterdam}}
    {\sc Rotterdam, The Netherlands}
  \\
  \subheaderrow
    {Guest Lecturer}
    {Jul \apo06 -- Jul \apo09}
  Yearly guest lectures on the phenomenon of open source, as part of the first year curriculum of \emph{Computer Science \& Economics}.

  \item  % LIP
  \headerrow
    {LIP Automatisering}
    {\sc Breda, The Netherlands}
  \\
  \subheaderrow
    {Software Auditor}
    {Sep \apo06}
  Audited their flag ship product \emph{\acr{LIP} Suite}:\ an \acr{ERP} solution for construction companies.

  \item  % The Health Agency
  \headerrow
    {\href{http://www.thehealthagency.com}{The Health Agency}}
    {\sc Delft \& Rotterdam, The Netherlands}
  \\
  \subheaderrow
    {Software Engineer}
    {Jun \apo05 -- Feb \apo06}
  {I worked on their innovative content management system (\acr{CMS}) targeted towards the health care sector. It is developed from scratch in Python and makes heavy use of Postgre\acr{SQL}, \acr{XML}/\acr{XSLT} and TwistedMatrix.
  \vspace{0.2em}}
  \subheaderrow
    {Software Auditor}
    {Dec \apo06}
  Assessed their Python/Zope/\acr{Z}o\acr{DB}-based web framework re-engineering project.

  \begin{center}
    \emph{Please refer to \href{http://www.linkedin.com/in/ciesbreijs}{my Linkedin profile} for the complete list of work experiences along with recommendations.}
  \end{center}

\end{itemize}
\vspace{-0.2em}

\hrule \vspace{-0.4em} \subsection*{Education}

\begin{itemize} \parskip=0.1em

  \item  % EUR
  \headerrow
    {Erasmus University Rotterdam}
    {\sc Rotterdam, The Netherlands}
  \subheaderrow
    {Bachelor degree in Computer Science \& Economics}
    {2004 -- 2007}
  Focussed on the economics of open source, rapid application development (\acr{RAD}) and the semantic web technology stack (\acr{RDF}/\acr{RDFS}, \acr{OWL} and \acr{SPARQL}).

  \item  % TUDelft
  \headerrow
    {Technical University Delft}
    {\sc Delft, The Netherlands}
  \\
  \subheaderrow
    {Industrial Design Engineering (discontinued)}
    {2001 -- 2002}

  \item  % Libanon
  \headerrow
    {Libanon Lyceum}
    {\sc Rotterdam, The Netherlands}
  \\
  \subheaderrow
    {\acr{VWO}}
    {1994 -- 2000}

\end{itemize}
\vspace{-0.3em}

\hrule \vspace{-0.4em} \subsection*{Skills}

  \begin{indentsection}{\parindent}  % Technical Secialties
  \begin{description*}
    \item[Technical specialties:]
    Web app development using Ruby (on Rails), Python and Java. Solid knowledge of web technologies:\ \acr{HTML}, \acr{CSS}, \acr{SQL}, \acr{XML}, \acr{RDF}, \acr{REST}, \acr{SOAP} and JavaScript (mainly jQuery). Linux administration skills:\ bash, Apache, My\acr{SQL}, Postgres\acr{SQL}, virtualization/cloud (Open\acr{VZ}, \acr{VM}ware, \acr{KVM}, Xen and \acr{EC}\capnums{2}), datacenter automation (Puppet and Chef), continuous integration (Hudson) and Ruby/\linebreak[0]\acr{PHP} web application stacks. Thorough \acr{\CPP} skills (mainly using Nokia's Qt lib for cross platform \acr{GUI} apps).
  \end{description*}
  \end{indentsection}

  \begin{indentsection}{\parindent}  % Natural Languages
  \begin{description*}
    \item[Natural languages:]
    Dutch \emph{(mother tongue)}, English \emph{(full professional proficiency)}, German \emph{(limited working proficiency)}, French \emph{(elementary proficiency)} and Mandarin Chinese \emph{(beginner)}.
  \end{description*}
  \end{indentsection}


\hrule \vspace{-0.4em} \subsection*{Interests}

  \begin{indentsection}{\parindent}
  \begin{description*}
    \item[Non-exhaustive and in alphabetical order:]
    art, Buddhism, cryptography, music, open source, philosophy, software engineering, travel, typography (e.g.\ graphic design, \LaTeX), \acr{UX} and vegetarian cooking.
  \end{description*}
  \end{indentsection}


\end{document}
